\section{结论}
本文作者对伪谱方法和分步 Fourier 方法进行数值实验(有限差分法的计算时间比较长,笔者没有对其进行完整的数值实验),比较这两种数值方法的计算时间和求解结果(实验结果见附录4)
\begin{enumerate}[label=(\arabic*)]
    \item 从求解的结果看,伪谱方法的结果更加准确和稳定,分步 Fourier 方法需要更小的空间和时间步长才能达到满意的结果;
    \item 从计算耗时来看,分步 Fourier 方法比伪谱方法更快,主要原因是伪谱方法必须通过求解频域的常微分方程,再进行 Fourier 逆变换才能得到时域结果。
\end{enumerate}
对三种数值方法求解单光孤子问题进行总结(见表\ref{tab:compare})
\begin{table}[htbp]
    \centering
    \caption{非线性 Schr\"odinger 方程的三种数值方法求解对比}
    \label{tab:compare}
    \begin{tabular}{cccc}
      \toprule 
        数值方法 & 计算类型 & 准确程度 & 计算时间 \\
      \midrule  
        伪谱方法 & 频域计算 & 准确 & 较快 \\
        分步 Fourier 方法 & 时域和频域交替计算 & 较准确 & 快 \\
        有限差分法 & 时域计算 & 较准确 & 非常慢 \\
      \bottomrule
    \end{tabular}
\end{table}

本文研究的问题是非线性 Schr\"odinger 方程的数值求解,着重研究单光孤子的周期性演化过程,在本文的最后对研究课题进行总结
\begin{enumerate}[label=(\arabic*)]
    \item 从解析方法看,非线性 Schr\"odinger 方程可以用 IST 方法求解;
    \item 从物理行为看,散射问题的离散本征值对应束缚态,其势场就是光孤子解;
    \item 从数值方法看,非线性 Schr\"odinger 方程可以用分步 Fourier 方法、伪谱方法、有限差分法进行求解。
\end{enumerate}
其实,非线性 Schr\"odinger 方程的研究方法、孤子概念都是从 KdV 方程迁移过来。就孤子问题来讲,还有很多有意思的课题,比如孤子稳定性、孤子相互作用等问题。