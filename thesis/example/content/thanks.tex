%!TEX root = ../GLUTthesis.tex
\section{致\quad 谢} % 无章节编号

% 公元二零二零年,岁次庚子,仲夏之月,吾之论文乃告杀青。辞穷理微,未敢称凌云之作,镂心鸟迹,得不效相如之叹?于是凭窗抱膝,寄情遐思。忆吾弱冠之龄入理工大学,意气方遒。尔来春秋有四,于今毕业,年齿已趋而立。户牅之外,万物滋荣,景致阙如昨日,堂室之内,吾已有苍颜白发矣。文凭两纸霜鬓两行,黄粱一枕功名一场,此皆寻常人生,乏善可陈。然吾身发受之父母,道德受之母校,学问受之师长,育教之恩,虽陨首结草不能报之万一。是以为情造文,铭而致谢。

大学四年即将结束,我也快要完成我的毕业论文,首先要感谢支持我完成论文的朋友。感谢我的指导老师王恒教授,给了我很大的发挥空间,在我对课题毫无头绪的时候指导我确定了具体的研究方向;感谢提供论文 \LaTeX 模板的胡光辉老师,他设计的论文模板让我不再被公式输入、排版等问题所困扰;感谢提供本文研究课题的巴灵丽同学,光孤子数值模拟问题来自她参加的夏令营的一个题目;感谢提供计算算力的裴俊杰同学,没有他的算力支持,我得不到有限差分的计算结果$\ldots\ldots$

其次要感谢生我养我的父母。我能来到这个世界这本身就是一件幸运的事,更何况是他们无微不至的关爱让我可以顺利健康地长大成人。小时候很淘气,给父母惹了很多麻烦;青春期比较叛逆,觉得父母老古董了,经常和他们对着干;长大了才发现,维持一个家庭,上有老,下有小是多么的不容易。父母正在老去,我要更加努力,为自己和社会创造更多价值,让他们不再操劳,能安度晚年。

最后要感谢陪伴我四年的桂林理工大学。感谢理学院的老师和同学,他们让我大学四年的学习生活不再孤单,经常给我提供帮助和建议以及挑战;感谢田径队的教练和队员,在大学因为共同的跑步爱好而聚在一起,互相鼓励、交流、提高,真的非常难得;感谢电子协会、数学建模协会和物理协会,让我大学四年的课余生活丰富多彩,社团活动给了我很多挑战,让我练出了非常多的技能$\ldots\ldots$即将离开母校,回想这四年,虽然有很多难熬的、烦心的事情,但是印象最深的还是那些奋斗过,快乐过的日子。
\newpage
