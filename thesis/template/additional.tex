%!TEX root = ../GLUTthesis.tex
\section{附录1} % 无章节编号
这里用来存放不应属于正文的内容、典型如代码、图纸、数学证明过程等内容。

物理常数表
\begin{table}[ht]
    \centering
 
    \begin{tabular}{lcll}
    \toprule 
          常数名称 & 符号 & 计算用值 & 单位 \\
    \midrule  

      真空中光速 & $c$ & $3.00\times 10^8$ & $\rm~ m\cdot s^{-1}$ \\
      
      万有引力常数 & $G$ & $6.67\times 10^{-11}$ & $\rm~ m^3\cdot s^{-2}$ \\
      
      阿伏伽德罗常数 & $N$ & $6.02\times 10^{23}$ & $\rm~ mol^{-1}$ \\

      玻尔兹曼常量 & $k$ & $1.38\times 10^{-23}$ & $\rm~ J\cdot K^{-1}$\\

      斯特藩-玻尔兹曼常数 & $\sigma$ & $5.67\times10^{-8}$ & $\rm~ W\cdot m^{-2}\cdot K^{-4}$\\ 
      
      理想气体摩尔体积 & $V_m$ & $22.4\times 10^{-3}$ & $\rm~ m^3\cdot mol^{-1}$ \\

      基本电荷 & $e$ & $1.60\times10^{-19}$ & $\rm~ C$\\

      电子静止质量 & $m_e$ & $9.11\times10^{-31}$ & $\rm~ kg$\\
      
      电子荷质比 & $e/m$ & $1.76\times10^{-11}$ & $\rm~ C\cdot kg^{-2}$\\

      质子静止质量 & $m_p$ & $1.67\times10^{-27}$ & $\rm~ kg$\\
      
      中字静止质量 & $m_n$ & $1.67\times10^{-27}$ & $\rm~ kg$\\
      
      $\alpha $粒子静止质量 &   & $6.645\times10^{-27}$ & $\rm~ kg$\\

      氢原子质量 & $m_\mathrm{H}$ & $1.67\times10^{-27}$ & $\rm~ kg$\\
      
      真空电导率 & $\varepsilon _0$ & $8.85\times10^{-12}$ & $\rm~ F\cdot m^{-2}$\\
      
      真空磁导率 & $\mu _0$ & $12.57\times10^{-7}$ & $\rm~ H\cdot m^{-1}$\\
      
      波尔半径 & $\alpha _0$ & $5.29\times10^{-11}$ & $\rm~ m$\\
      
      真空磁子 & $\mu _B$ & $9.27\times10^{-24}$ & $\rm~ J\cdot T^{-1}$\\
      
      普朗克常量 & $h$ & $6.63\times10^{-34}$ & $\rm~ J\cdot s$\\
    \bottomrule
    \end{tabular}
\end{table}

\newpage
%%%%%%%%%%%%%%%%%%%%%%%%%%%%%%%%%
\section{附录2} 


这是本设计所用的源程序代码:

\begin{lstlisting}[language={C}, caption={}]
/*程序:ex1_1.c */
/*功能:控制一个LED闪烁程序 */
#include <reg51.h>  /*包含头文件reg51.h,定义了51单片机的专用寄存器*/
sbit P1_0=P1^0;          	/*定义位名称*/
/*函数名:delay*/
/*函数功能:实现软件延时*/
/*形式参数:无符号整型变量i,控制空循环的循环次数*/
/*返回值:无
void delay(unsigned int i) 	/*延时函数*/
{
  unsigned int k;
	for(k=0;k<i;k++);
}
void main()                	/*主函数*/
{
  while(1){
     P1_0=0;				/*点亮LED*/
     delay(10000);			/*调用延时函数,实际参数为10000*/
     P1_0=1;				/*熄灭LED*/
     delay(10000);			/*调用延时函数,实际参数为10000*/
           }
 }

\end{lstlisting}
